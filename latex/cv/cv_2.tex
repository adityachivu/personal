%%%%%%%%%%%%%%%%%%%%%%%%%%%%%%%%%%%%%%%%%
% Medium Length Graduate Curriculum Vitae
% LaTeX Template
% Version 1.1 (9/12/12)
%
% This template has been downloaded from:
% http://www.LaTeXTemplates.com
%
% Original author:
% Rensselaer Polytechnic Institute (http://www.rpi.edu/dept/arc/training/latex/resumes/)
%
% Important note:
% This template requires the res.cls file to be in the same directory as the
% .tex file. The res.cls file provides the resume style used for structuring the
% document.
%
%%%%%%%%%%%%%%%%%%%%%%%%%%%%%%%%%%%%%%%%%

%----------------------------------------------------------------------------------------
%	PACKAGES AND OTHER DOCUMENT CONFIGURATIONS
%----------------------------------------------------------------------------------------

\documentclass[margin, 10pt]{res} % Use the res.cls style, the font size can be changed to 11pt or 12pt here

\usepackage{hyperref} % For including hyperlinks
\usepackage{helvet} % Default font is the helvetica postscript font
\hypersetup{
    colorlinks=true,
    linkcolor=blue,
    filecolor=magenta,      
    urlcolor=blue,
}
%\usepackage{newcent} % To change the default font to the new century schoolbook postscript font uncomment this line and comment the one above

\setlength{\textwidth}{5.1in} % Text width of the document

\begin{document}

%----------------------------------------------------------------------------------------
%	NAME AND ADDRESS SECTION
%----------------------------------------------------------------------------------------

\moveleft.5\hoffset\centerline{\large\bf Adityakrishna Chivukula} % Your name at the top
 
\moveleft\hoffset\vbox{\hrule width\resumewidth height 1pt}\smallskip % Horizontal line after name; adjust line thickness by changing the '1pt'
 
\moveleft.5\hoffset\centerline{Dept. of Computer Science and Information Systems} % Your address
\moveleft.5\hoffset\centerline{BITS Pilani, Hyderabad Campus}
\moveleft.5\hoffset\centerline{aditya.chivukula@gmail.com}

%----------------------------------------------------------------------------------------

\begin{resume}

%----------------------------------------------------------------------------------------
%	EDUCATION SECTION
%----------------------------------------------------------------------------------------

\section{EDUCATION}

{\sl \textbf{BITS Pilani, Hyderabad, India}}  \\
B.E. (Honours), Computer Science. Discipline GPA: 8.13/10\hfill expected, July 2017 \\ 
{\sl \textbf{AECS Maaruti Magnolia Public School}}  \\
Senior Secondary School, Science Stream. AISSE: 93.6\% \hfill expected, May 2013 \\ 
 
%----------------------------------------------------------------------------------------
%	PUBLCIATIONS SECTION
%----------------------------------------------------------------------------------------

\section{PUBLICATIONS} 
{\sl \textbf{On the Modeling of Error Functions as High Dimensional Landscapes for Weight Initialization in Learning Networks,}} \\ Julius, Gopinath Mahale, Sumana T., Adityakrishna Chivukula\\In the proceedings of International Conference on Embedded Systems, Architecture, Modeling and Simulation (SAMOS XVI) \href{https://arxiv.org/abs/1607.06011}{[arXiv]}
 
%----------------------------------------------------------------------------------------
%	ACADEMIC PROJECTS SECTION
%----------------------------------------------------------------------------------------
 
\section{ACADEMIC \\ PROJECTS}

{\sl \textbf{Deep Learning, Techniques and Applications}} \hfill April 2016 - Present\\
Studying the various Deep Learning techniques including
\begin{itemize} \itemsep -2pt % Reduce space between items
\item \textit{Network Architecture: } CNN, RNN, LSTM, GAN, Autoencoders
\item \textit{Optimization: } Stochastic Gradient Descent, RMSprop, Adagrad and variants, second-order optimizations techniques.
\item \textit{Implementation: } Code primarily written in Tensorflow and run on standard datasets (CIFAR, ImageNet). Other libraries used were Caffe and Torch.
\end{itemize}
 
{\sl \textbf{Augmented Reality using Deep Learning}} \\
{\sl \textbf{for Scene Understanding}} \hfill April 2016 - Present\\
Design Deep Learning system for real-time semantic segmentation and labelling of key objects in view. Some of the papers that inspired this project are:
\begin{itemize} \itemsep -2pt % Reduce space between items
\item {\sl Learning to Segment Object Candidates}, Pinheiro et al.
\item {\sl Conditional Random Fields as Recurrent Neural Networks}, Zeng et al.
\end{itemize} 

{\sl \textbf{Automatic Differentiation}} \hfill Jan 2016 - May 2016 \\
Worked on implementing a simplistic operator overloading based reverse mode AD tool. Specifically, I implemented DAGs using a custom AD tool with composite functions represented as independant gates rather than a combination of basic operators. Studied multiple taping methods and DAG reductions.

{\sl \textbf{Real-Time Rainfall Simulation in OpenGL}} \hfill Aug 2015 - Dec 2015 \\
Implemented a physics-based droplet rendering rainfall simulation based on the approach mentioned in the following paper:\\
{\sl Real-time Rain Simulation in Cartoon Style}, Feng et al \href{https://arxiv.org/abs/1607.06011}{[GitHub]}

%----------------------------------------------------------------------------------------
%	RESEARCH INTERNSHIPS SECTION
%----------------------------------------------------------------------------------------
 
\section{RESEARCH \\ INTERNSHIPS}
{\sl \textbf{CADLab, Indian Institute of Science, Bangalore}} \hfill Dec 2015 - Mar 2016 \\
Worked on a Statistical Physics formulation of the optimization objective for Neural Networks. Using results from Random Matrix Theory for Spin Glass Models, we define a weight initialisation that performs better than current initialisation methods on simple networks. Results published in SAMOS XVI.

%----------------------------------------------------------------------------------------
%	TEACHING SECTION
%----------------------------------------------------------------------------------------
 
\section{TEACHING}
{\sl \textbf{Deep Learning Society}} \hfill Aug 2016 - Present\\
Conduct weekly discussion sessions on trending Deep Learning and their applications in practice. Topics include the mathematical formulation of a variety of networks and optimization objectives, methods of implementation and current research trends.

{\sl \textbf{Teaching Assistant}} \hfill Aug 2015 - Dec 2015\\
CS F222, Discrete Sructures for Computer Science
\begin{itemize}  \itemsep -2pt % Reduce space between items
\item Designed coding assignment questions for evalutaion.
\item Implemented portal for submitting and evaluating responses, hosted on the intranet.
\end{itemize}
%----------------------------------------------------------------------------------------
%	SKILLS SECTION
%----------------------------------------------------------------------------------------

\section{SKILLS} 

\begin{itemize}
\item {\sl Languages \& Software:} {(\it in order of proficiency}) \\
 C++, Python, MATLAB/Octave, Java, R
\item {\sl Libraries:} Tensorflow, Caffe, Torch, OpenGL, CGAL, OpenCV.
\end{itemize}

%----------------------------------------------------------------------------------------
%	EXTRA-CURRICULAR ACTIVITIES SECTION
%----------------------------------------------------------------------------------------

\section{EXTRA-CURRICULAR \\ ACTIVITIES} 

\textbf{Head of Editorial Board, BITS Embryo} \\
BITS Embryo is a student body that invites renowned speakers, domain experts and successful alumni from around the world enagaged in a variety of domains to deliver guest lectures for the students of the university.\\ \\
\textbf{ Volunteer, HelpAge India - Bangalore} \\
HelpAge India is a nation-wide NGO that works towards assisting and improving the well-being of senior citizens.

%----------------------------------------------------------------------------------------
%	REFERENCES SECTION
%----------------------------------------------------------------------------------------

\section{REFERENCES} 

\textbf{Prof. S.K.Nandy} \\
Convener, CADLab \\
Dept. of Computational and Data Sciences \\
Indian Institute of Science, Bangalore

%----------------------------------------------------------------------------------------

\end{resume}
\end{document}